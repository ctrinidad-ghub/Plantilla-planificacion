\documentclass[11pt]{charter}

% El títulos de la memoria, se usa en la carátula y se puede usar el cualquier lugar del documento con el comando \ttitle
\titulo{Medición y aceptación de parámetros en transformadores de corriente alterna} 

% Nombre del posgrado, se usa en la carátula y se puede usar el cualquier lugar del documento con el comando \degreename
\posgrado{Carrera de Especialización en Sistemas Embebidos} 
%\posgrado{Carrera de Especialización en Internet de las Cosas} 
%\posgrado{Carrera de Especialización en Intelegencia Artificial}
%\posgrado{Maestría en Sistemas Embebidos} 
%\posgrado{Maestría en Internet de las cosas}

% Tu nombre, se puede usar el cualquier lugar del documento con el comando \authorname
\autor{Cristian Trinidad} 

% El nombre del director y co-director, se puede usar el cualquier lugar del documento con el comando \supname y \cosupname y \pertesupname y \pertecosupname
\director{Nombre del Director}
\pertenenciaDirector{pertenencia} 
% FIXME:NO IMPLEMENTADO EL CODIRECTOR ni su pertenencia
\codirector{} % si queda vacio no se deberíá incluir 
\pertenenciaCoDirector{}

% Nombre del cliente, quien va a aprobar los resultados del proyecto, se puede usar con el comando \clientename y \empclientename
\cliente{Esp. Lic. Leopoldo A. Zimperz}
\empresaCliente{Iris Tecnologia S.R.L.}

% Nombre y pertenencia de los jurados, se pueden usar el cualquier lugar del documento con el comando \jurunoname, \jurdosname y \jurtresname y \perteunoname, \pertedosname y \pertetresname.
\juradoUno{Nombre y Apellido (1)}
\pertenenciaJurUno{pertenencia (1)} 
\juradoDos{Nombre y Apellido (2)}
\pertenenciaJurDos{pertenencia (2)}
\juradoTres{Nombre y Apellido (3)}
\pertenenciaJurTres{pertenencia (3)}
 
\fechaINICIO{22 de junio de 2020}		%Fecha de inicio de la cursada de GdP \fechaInicioName
\fechaFINALPlanificacion{22 de Agosto de 2020} 	%Fecha de final de cursada de GdP
\fechaFINALTrabajo{22 de diciembre de 2020}		%Fecha de defensa pública del trabajo final


\begin{document}

\maketitle
\thispagestyle{empty}
\pagebreak


\thispagestyle{empty}
{\setlength{\parskip}{0pt}
\tableofcontents{}
}
\pagebreak


\section{Registros de cambios}
\label{sec:registro}


\begin{table}[ht]
\label{tab:registro}
\centering

\begin{tabularx}{\linewidth}{@{}|c|X|c|@{}}
\hline
\rowcolor[HTML]{C0C0C0} 
Revisión & \multicolumn{1}{c|}{\cellcolor[HTML]{C0C0C0}Detalles de los cambios realizados} & Fecha      \\ \hline
1.0      & Creación del documento                                                          & 22/06/2020 \\ \hline
1.1      & Ejemplo de un texto muy largo que debiera ocupar más de una línea para que tengan de ejemplo                                                                                																						   & dd/mm/aaaa \\ \hline
1.2      & Otro ejemplo \newline
		   Con texto partido \newline
		   En varias líneas \newline
		   A propósito                                                                     & dd/mm/aaaa \\ \hline
\end{tabularx}
\end{table}

\pagebreak



\section{Acta de Constitución del Proyecto}
\label{sec:acta}

\begin{flushright}
Buenos Aires, \fechaInicioName
\end{flushright}

\vspace{2cm}

Por medio de la presente se acuerda con el Ing. \authorname\hspace{1px} que su Trabajo Final de la \degreename\hspace{1px} se titulará ``\ttitle'', consistirá esencialmente en el prototipo preliminar de un dispositivo capaz de medir y registrar parámetros de transformadores de corriente alterna empleados en la fabricación de dispositivos electromédicos, y tendrá un presupuesto preliminar estimado de 600 hs de trabajo y \textcolor{red}{\$XXX}, con fecha de inicio \fechaInicioName\hspace{1px} y fecha de presentación pública \fechaFinalName.

Se adjunta a esta acta la planificación inicial.

\vfill

% Esta parte se construye sola con la información que hayan cargado en el preámbulo del documento y no debe modificarla
\begin{table}[ht]
\centering
\begin{tabular}{ccc}
\begin{tabular}[c]{@{}c@{}}Ariel Lutenberg \\ Director posgrado FIUBA\end{tabular} &  & \begin{tabular}[c]{@{}c@{}}\clientename \\ \empclientename \end{tabular} \vspace{2.5cm} \\ 
\multicolumn{3}{c}{\begin{tabular}[c]{@{}c@{}} \supname \\ Director del Trabajo Final\end{tabular}} \vspace{2.5cm} \\
\begin{tabular}[c]{@{}c@{}}\jurunoname \\ Jurado del Trabajo Final\end{tabular}     &  & \begin{tabular}[c]{@{}c@{}}\jurdosname\\ Jurado del Trabajo Final\end{tabular}  \vspace{2.5cm}  \\
\multicolumn{3}{c}{\begin{tabular}[c]{@{}c@{}} \jurtresname\\ Jurado del Trabajo Final\end{tabular}} \vspace{.5cm}                                                                     
\end{tabular}
\end{table}




\section{Descripción técnica-conceptual del Proyecto a realizar}
\label{sec:descripcion}


La empresa Iris tecnología S.R.L. fabrica equipos electromédicos cumpliendo con los requisitos de “Buenas Prácticas de Fabricación” exigidos por la Administración Nacional de Medicamentos, Alimentos y Tecnología Médica (A.N.M.A.T.). Adicionalmente incorporó un sistema de gestión de calidad para fabricación de dispositivos médicos, basado en la norma ISO13485:2016. En la actualidad se está trabajando en la incorporación de tecnologías para agilizar tareas y controles que deben realizarse acorde a dicho sistema de calidad. 

Entre las tareas y controles antes mencionados se encuentra el ensayo y calificación de  transformadores para su posterior utilización en las líneas de producción. El objetivo de este proyecto es proveer un dispositivo capaz de automatizar este procedimiento, el cual, actualmente se realiza en forma manual. Esta labor además de insumir una gran cantidad de tiempo y ser muy repetitiva, presenta un gran riesgo de error humano, pudiendo además, comprometer la seguridad del operario y la fiabilidad de los datos obtenidos.

La idea es desarrollar un prototipo capaz de medir y registrar los parámetros de transformadores de tensión alterna empleados en la fabricación de dispositivos electromédicos, a fin ser aceptados o rechazados para su posterior uso. Los parámetros a medir incluyen las tensiones y corrientes de los bobinados primario y secundario.

En la Figura \ref{fig:diagBloques} se presenta el diagrama en bloques del sistema.

\vspace{25px}

\begin{figure}[htpb]
\centering 
\includegraphics[width=.7\textwidth]{./Figuras/diagBloques.png}
\caption{Diagrama en bloques del sistema}
\label{fig:diagBloques}
\end{figure}

\vspace{25px}


El dispositivo propuesto deberá indicar al operador si el transformador es aceptado o rechazado en base a umbrales previamente configurados por medio de una comunicación WiFi. Además de mostrar el resultado del ensayo localmente, este y las mediciones realizadas deberán ser enviados a un servidor web a través de la comunicación WiFi y adicionalmente se debe imprimir una etiqueta que permita la trazabilidad del componente.


A continuación se presenta un resumen de las tareas propuestas para el dispositivo a desarrollar:
\begin{itemize}
\item Mediciones de tensión y corriente a los transformadores bajo ensayo.
\item Configuración de umbrales de aceptación para los parámetros medidos.
\item Emitir confirmación de aceptación o rechazo del transformador.
\item Visualización en display LCD de umbrales seteados y mediciones.
\item Envío de las mediciones a servidor web, vía red WiFi.
\item Impresión de etiquetas con número de partida y valores medidos. 
\end{itemize}


\section{Identificación y análisis de los interesados}
\label{sec:interesados}

\begin{consigna}{red} 
\begin{table}[ht]
%\caption{Identificación de los interesados}
%\label{tab:interesados}
\begin{tabularx}{\linewidth}{@{}|l|X|X|l|@{}}
\hline
\rowcolor[HTML]{C0C0C0} 
Rol           & Nombre y Apellido & Organización 	& Puesto 	\\ \hline
Cliente       & \clientename      &\empclientename	& Socio     \\ \hline
Responsable   & \authorname       & FIUBA        	& Alumno 	\\ \hline
Orientador    & \supname	      & \pertesupname 	& Director	Trabajo final \\ \hline
Equipo        & \authorname       & FIUBA        	& Alumno	\\ \hline
Usuario final & Operadores        &\empclientename	& Operador 	\\ \hline
\end{tabularx}
\end{table}
\end{consigna}



\section{1. Propósito del proyecto}
\label{sec:proposito}

El propósito de este proyecto es disminuir el riesgo de error humano y aumentar la fiabilidad de los datos obtenidos al evaluar los transformadores. Así también como aumentar la seguridad de los operadores expuestos a altas tensiones al simplificar el procedimiento de caracterización e incorporar métodos de protección en el prototipo a diseñar.


\section{2. Alcance del proyecto}
\label{sec:alcance}

El presente proyecto, tiene como alcance:
\begin{itemize}
\item Análisis, investigación y elección del hardware.
\item Investigar sobre un modelo de impresora a adquirir, cuyo protocolo esté disponible y permita llevar a cabo el proyecto.
\item Diseño e implementación del firmware del sistema en lenguaje C.
\item Desarrollo de un prototipo funcional sobre un PCB tipo universal.
\item Elaboración de un manual de usuario con la información mínima requerida para la operación del dispositivo y con las recomendaciones para evitar descargas de alta tensión. 
\end{itemize}

Queda excluido del alcance de este proyecto:
\begin{itemize}
\item El desarrollo de circuitos de medición de tensión y/o corriente alterna de precisión. Se acepta la precisión obtenida de módulos comerciales como el ZMPT101B.
\item El desarrollo de fuentes de tensión alterna para alimentar los transformadores en ensayo. El cliente deberá proporcionar al dispositivo los 230Vac estabilizados necesarios para alimentar el bobinado primario y la tensión alterna adecuada para alimentar el bobinado secundario.
\item El desarrollo de una interfaz web desde la cual interactuar por WiFi con el módulo.
\item El desarrollo del servidor web o API de colección de datos del en el servidor/PC del cliente. Ésta será provista por el cliente.
\item El desarrollo de un prototipo de fabricación escalable que cumpla con todas las certificaciones necesarias.
\item El desarrollo de un PCB, más allá del prototipo en placa universal.
\item El diseño del gabinete para la instalación del sistema electrónico. El dispositivo se proveerá en algún tipo de gabinete, pero de ninguna manera se pretende un desarrollo detallado de éste.
\item El diseño de circuitos de protección del operario contra descargas eléctricas de alta tensión. En este sentido se asume que el operario que utilizará el dispositivo es idóneo en el tema y que el cliente posee en sus instalaciones las medidas de seguridad pertinentes para el trabajo con altas tensiones como disyuntores y puestas a tierras acorde con la normativa vigente de Higiene y Seguridad en el Trabajo.
\end{itemize}

\section{3. Supuestos del proyecto}
\label{sec:supuestos}


Para el desarrollo del presente proyecto se supone que:

\begin{itemize}
\item La precisión necesaria para las mediciones de tensión y/o corrientes pueden ser obtenida con el uso de módulos comerciales como el ZMPT101B con modificaciones mínimas.
\item Los costos de los materiales son cubiertos por el cliente.
\item Los materiales necesarios para el armado del prototipo pueden ser comprados en el país, aun con las restricciones de importación impuestas por el COVID-19.
\item El cliente proveerá los transformadores a medir.
\item El cliente proveerá la impresora a utilizar.
\item La impresora a utilizar posee interfaz RS-232.
\item El protocolo de la impresora a utilizar debe estar disponible para su implementación. No está contemplado dentro del proyecto hacer ingeniería inversa sobre la impresora a utilizar.
\item Si bien se incluyen algunas protecciones para el manejo de tensiones de 220 Vac, como se dijo anteriormente, se asume que el operario es idóneo y conoce los riesgos en la operación del dispositivo y que el cliente cumple las reglamentaciones de Higiene y Seguridad en el Trabajo.
\item El cliente es responsable de proporcionar, en caso de ser necesario y de una manera simple de entender e implementar, los requerimientos asociados al cumplimiento de su sistema de gestión de calidad para la fabricación de dispositivos médicos, basado en la norma ISO13485:2016 y cualquier otra normativa de cumplimiento necesario.
\end{itemize}



\section{4. Requerimientos}
\label{sec:requerimientos}


\begin{enumerate}
\item Requerimientos de hardware
	\begin{enumerate}
	\item El dispositivo debe ser diseñado basado en el módulo ESP32.
	\item El dispositivo debe ser capaz de medir tensiones y corrientes alternas de manera aislada del transformador bajo ensayo.
	\item El dispositivo debe ser capaz de conmutar las tensiones primarias y secundarias.
	\item El dispositivo debe tener un display LCD alfanumérico 20x4 caracteres.
	\item El dispositivo debe tener una interfaz Wifi.
	\item El dispositivo debe tener una interfaz RS232 capaz de manejar impresoras.
	\item El dispositivo debe alimentarse desde la red eléctrica Argentina 220Vac/50hz, debiendo proveerse todas las alimentaciones a los diferentes módulos de hardware.
	\item El dispositivo debe poseer un switch para indicar que la tapa de seguridad ha sido removida.
	\item El dispositivo debe poseer 3 pulsadores nombrados Configurar, Testear y Cancelar.
	\item El dispositivo debe poseer un buzzer.
	\end{enumerate}
\item Requerimientos relativos a los valores a medir
	\begin{enumerate}
	\item Bobinado primario
		\begin{enumerate}
		\item Rango tensión: 100 – 240Vac
		\item Rango corriente: 0 – 800 mAac
		\end{enumerate}
	\item Bobinado secundario
		\begin{enumerate}
		\item Rango tensión: 0 – 30 Vac
		\item Rango corriente: 0 – 1500 mAac
		\end{enumerate}
	\item Precisión en la medición de tensión: mejor al 1.5\% (puede variar en base a lo que se pueda conseguir en el mercado)
	\item Precisión en la medición de corriente: mejor al 4\% (puede variar en base a lo que se pueda conseguir en el mercado)
	\end{enumerate}
\item Requerimientos funcionales
	\begin{enumerate}
	\item El dispositivo debe permitir que se configuren los umbrales mínimos y máximos para los parámetros medidos. 
	\item Los valores a configurar deben ser adquiridos solo por WiFi, no se admite configuración local por teclado.
	\item El dispositivo debe generar una confirmación de aceptación o rechazo del transformador en ensayo basado en los umbrales mínimos y máximos preseteados.
	\item El dispositivo, luego de cada ensayo, independiente del resultado, debe imprimir etiquetas con la siguiente información:
		\begin{enumerate}
		\item Número de partida del transformador ensayado
		\item Condición de aceptado o rechazado
		\item Valores medidos de tensiones y corrientes
		\end{enumerate}
	\item El dispositivo debe seguir de forma automática la siguiente secuencia de testeo basico:
		\begin{itemize}
		\item Pulsar Testear
		\item Verificar si el dispositivo fue configurado, sino pedir parámetros de configuración
		\item Energizar el bobinado primario con 230Vac estabilizados  (provistos externamente por el cliente)
		\item Medir:
			\begin{itemize}
			\item Tensión en bobinado primario
			\item Corriente que circula por el bobinado primario
			\item Tensión en bobinado secundario
			\end{itemize}
		\item Desenergizar el bobinado primario
		\item Energizar el bobinado secundario con la tensión adecuada (provista externamente por el cliente)
		\item Medir:
			\begin{itemize}
			\item Tensión en bobinado primario
			\item Tensión en bobinado secundario
			\item Corriente que circula por el bobinado secundario
			\end{itemize}
		\item Desenergizar el bobinado secundario
		\item Comparar los valores medidos con los umbrales configurados previamente y determinar si el transformador es aceptado o rechazado
		\end{itemize}
	\end{enumerate}	
\item Requerimientos de comunicación (interfaz WiFi)
	\begin{enumerate}
	\item Solicitar parámetros de configuración: el dispositivo debe generar un comando GET de http a un servidor web (provisto por el cliente) para obtener los umbrales de aceptación y el número de partida del transformador a ensayar.
	\item El dispositivo de ser capaz de procesar la respuesta del comando GET que estará en formato JSON.
	\item Enviar resultados del ensayo: el dispositivo debe generar un comando POST de http a un servidor (provisto por el cliente) con la información del transformador ensayado en formato JSON.
	\end{enumerate}
\item Requerimientos de interfaz de usuario
	\begin{enumerate}
	\item Sobre la funcionalidad de los pulsadores
		\begin{enumerate}
		\item Configurar: al pulsar este pulsador el dispositivo debe generar el comando GET para solicitar al servidor web los parámetros de configuración del dispositivo a través de WiFi.
		\item Testear: al pulsar este pulsador el dispositivo debe iniciar la secuencia de testeo.
		\item Cancelar: al pulsar este pulsador la secuencia de testeo en curso debe quedar abortada.
		\end{enumerate}
	\item Sobre la funcionalidad del display LCD
		\begin{enumerate}
		\item El dispositivo debe mostrar los umbrales presetedos y los valores de las mediciones actuales.
		\item Los valores de los umbrales configurados en el dispositivo deberán permanecer en el display  durante el ensayo.
		\item El dispositivo debe mostrar los valores medidos del transformador ensayado luego de cada medición.
		\item Luego de finalizado el ensayo se debe mostrar un mensaje que indica que la información del ensayo se ha enviado al servidor web y mantenerse el dispositivo bloqueado hasta que se haya recibido la respuesta del servidor.
		\end{enumerate}
	\item Sobre el buzzer
		\begin{enumerate}
		\item El dispositivo debe emitir un solo sonido de 0,5 segundos de duración para confirmar la aceptación del transformador.
		\item El dispositivo debe emitir un sonido de repetición de 5 ciclos de 0,5 segundos de duración con pausas de 0,5 segundos para confirmar el rechazo del transformador.
		\end{enumerate}		
	\end{enumerate}
\end{enumerate}





\section{5. Entregables principales del proyecto}
\label{sec:entregables}


\begin{itemize}
\item Lista de componentes
\item Diagrama esquemático de la placa universal
\item Diagrama de cableado del dispositivo
\item Código fuente en C
\item Prototipo del equipo funcionando
\item Manual de uso
\item Informe final
\end{itemize}



\section{6. Desglose del trabajo en tareas}
\label{sec:wbs}


\begin{enumerate}
\item Análisis y definición de requerimientos (40 hs)
	\begin{enumerate}
	\item Definición del alcance y captura de requerimientos con el cliente (20 hs).
	\item Análisis de factibilidad: investigación sobre características de sensores de corriente alterna disponibles en el mercado (10 hs).
	\item Planificación del proyecto, escritura de documentos previos (10 hs).
	\end{enumerate}
\item Desarrollo de hardware (170 hs)
	\begin{enumerate}
	\item Calculo, simulación y selección de los componentes de hardware (45 hs).
	\item Diseño del diagrama esquemático (25 hs).
	\item Gestión de compra de componentes con el cliente (15 hs).
	\item Armado de prototipo de hardware preliminar en protoboard (20 hs).
	\item Armado del circuito final en placa universal (40 hs).
	\item Ensamblado del prototipo (15 hs).
	\item Pruebas preliminares de hardware (10 hs).
	\end{enumerate}
\item Desarrollo de software (240 hs)
	\begin{enumerate}
	\item Preparación del entorno de trabajo, descarga e instalación de IDE y puesta punto del programador (5 hs).
	\item Diseño de la arquitectura de software (20 hs).
	\item Diseño de rutinas de medición (verdadero valor eficaz) (25 hs).
	\item Diseño de rutinas funcionales (40 hs).
	\item Diseño de rutinas de comunicación WiFi (40 hs).
	\item Desarrollo de la interfaz de usuario (40 hs).
	\item Investigación y desarrollo de librerías para el manejo de la impresora (30 hs).
	\item Integración de las tareas en el RTOS (40 hs).
	\end{enumerate}
\item Verificación y validación del dispositivo (60 hs)
	\begin{enumerate}
	\item Desarrollo de pruebas unitarias (20 hs).
	\item Desarrollo de pruebas de integración de los componentes de hardware/software (20hs).
	\item Desarrollo de pruebas funcionales con diferentes transformadores (20 hs).
	\end{enumerate}
\item Cierre del proyecto (100 hs)
	\begin{enumerate}
	\item  Informes de avance del proyecto (10 hs).
	\item  Confección del manual de usuario (20 hs).
	\item  Memoria descriptiva final del proyecto (50 hs).
	\item  Elaboración de presentación final (20 hs).
	\end{enumerate}		
\end{enumerate}

Cantidad total de horas: (610 hs)



\section{7. Diagrama de Activity On Node}
\label{sec:AoN}

\begin{consigna}{red}
Armar el AoN a partir del WBS definido en la etapa anterior. 

%La figura \ref{fig:AoN} fue elaborada con el paquete latex tikz y pueden consultar la siguiente referencia \textit{online}:

%\url{https://www.overleaf.com/learn/latex/LaTeX_Graphics_using_TikZ:_A_Tutorial_for_Beginners_(Part_3)\%E2\%80\%94Creating_Flowcharts}

\end{consigna}

\begin{figure}[htpb]
\centering 
\includegraphics[width=.8\textwidth]{./Figuras/AoN.png}
\caption{Diagrama en \textit{Activity on Node}}
\label{fig:AoN}
\end{figure}

Indicar claramente en qué unidades están expresados los tiempos.
De ser necesario indicar los caminos semicríticos y analizar sus tiempos mediante un cuadro.
Es recomendable usar colores y un cuadro indicativo describiendo qué representa cada color, como se muestra en el siguiente ejemplo:



\section{8. Diagrama de Gantt}
\label{sec:gantt}

\begin{consigna}{red}
Utilizar el software Gantter for Google Drive o alguno similar para dibujar el diagrama de Gantt.

Existen muchos programas y recursos \textit{online} para hacer diagramas de gantt, entre las cuales destacamos:

\begin{itemize}
\item Planner
\item GanttProject
\item Trello + \textit{plugins}. En el siguiente link hay un tutorial oficial: \\ \url{https://blog.trello.com/es/diagrama-de-gantt-de-un-proyecto}
\item Creately, herramienta online colaborativa. \\\url{https://creately.com/diagram/example/ieb3p3ml/LaTeX}
\item Se puede hacer en latex con el paquete \textit{pgfgantt}\\ \url{http://ctan.dcc.uchile.cl/graphics/pgf/contrib/pgfgantt/pgfgantt.pdf}
\end{itemize}

Pegar acá una captura de pantalla del diagrama de Gantt, cuidando que la letra sea suficientemente grande como para ser legible. 
Si el diagrama queda demasiado ancho, se puede pegar primero la ``tabla'' del Gantt y luego pegar la parte del diagrama de barras del diagrama de Gantt.

Configurar el software para que en la parte de la tabla muestre los códigos del EDT (WBS).\\
Configurar el software para que al lado de cada barra muestre el nombre de cada tarea.\\
Revisar que la fecha de finalización coincida con lo indicado en el Acta Constitutiva.

En la figura \ref{fig:gantt}, se muestra un ejemplo de diagrama de gantt realizado con el paquete de \textit{pgfgantt}. En la plantilla pueden ver el código que lo genera y usarlo de base para construir el propio.

\begin{figure}[htbp]
\begin{center}
\begin{ganttchart}{1}{12}
  \gantttitle{2020}{12} \\
  \gantttitlelist{1,...,12}{1} \\
  \ganttgroup{Group 1}{1}{7} \\
  \ganttbar{Task 1}{1}{2} \\
  \ganttlinkedbar{Task 2}{3}{7} \ganttnewline
  \ganttmilestone{Milestone o hito}{7} \ganttnewline
  \ganttbar{Final Task}{8}{12}
  \ganttlink{elem2}{elem3}
  \ganttlink{elem3}{elem4}
\end{ganttchart}
\end{center}
\caption{Diagrama de gantt de ejemplo}
\label{fig:gantt}
\end{figure}

\end{consigna}

\section{9. Matriz de uso de recursos de materiales}
\label{sec:recursos}


\begin{table}[htpb]
\label{tab:recursos}
\centering
\begin{tabularx}{\linewidth}{@{}|c|X|X|X|X|X|@{}}
\hline
\cellcolor[HTML]{C0C0C0} & \cellcolor[HTML]{C0C0C0} & \multicolumn{4}{c|}{\cellcolor[HTML]{C0C0C0}Recursos requeridos (horas)} \\ \cline{3-6} 
\multirow{-2}{*}{\cellcolor[HTML]{C0C0C0}\begin{tabular}[c]{@{}c@{}}Código\\ WBS\end{tabular}} & \multirow{-2}{*}{\cellcolor[HTML]{C0C0C0}\begin{tabular}[c]{@{}c@{}}Nombre \\ tarea\end{tabular}} & Material 1 & Material 2 & Material 3 & Material 4 \\ \hline
 &  &  &  &  &  \\ \hline
 &  &  &  &  &  \\ \hline
 &  &  &  &  &  \\ \hline
 &  &  &  &  &  \\ \hline
\end{tabularx}%
\end{table}


\section{10. Presupuesto detallado del proyecto}
\label{sec:presupuesto}

\begin{consigna}{red}
Si el proyecto es complejo entonces separarlo en partes:
\begin{itemize}
\item Un total global, indicando el subtotal acumulado por cada una de las áreas.
\item El desglose detallado del subtotal de cada una de las áreas.
\end{itemize}

IMPORTANTE: No olvidarse de considerar los COSTOS INDIRECTOS.

\end{consigna}

\begin{table}[htpb]
\centering
\begin{tabularx}{\linewidth}{@{}|X|c|r|r|@{}}
\hline
\rowcolor[HTML]{C0C0C0} 
\multicolumn{4}{|c|}{\cellcolor[HTML]{C0C0C0}COSTOS DIRECTOS} \\ \hline
\rowcolor[HTML]{C0C0C0} 
Descripción &
  \multicolumn{1}{c|}{\cellcolor[HTML]{C0C0C0}Cantidad} &
  \multicolumn{1}{c|}{\cellcolor[HTML]{C0C0C0}Valor unitario} &
  \multicolumn{1}{c|}{\cellcolor[HTML]{C0C0C0}Valor total} \\ \hline
 &
  \multicolumn{1}{c|}{} &
  \multicolumn{1}{c|}{} &
  \multicolumn{1}{c|}{} \\ \hline
 &
  \multicolumn{1}{c|}{} &
  \multicolumn{1}{c|}{} &
  \multicolumn{1}{c|}{} \\ \hline
\multicolumn{1}{|l|}{} &
   &
   &
   \\ \hline
\multicolumn{1}{|l|}{} &
   &
   &
   \\ \hline
\multicolumn{3}{|c|}{SUBTOTAL} &
  \multicolumn{1}{c|}{} \\ \hline
\rowcolor[HTML]{C0C0C0} 
\multicolumn{4}{|c|}{\cellcolor[HTML]{C0C0C0}COSTOS INDIRECTOS} \\ \hline
\rowcolor[HTML]{C0C0C0} 
Descripción &
  \multicolumn{1}{c|}{\cellcolor[HTML]{C0C0C0}Cantidad} &
  \multicolumn{1}{c|}{\cellcolor[HTML]{C0C0C0}Valor unitario} &
  \multicolumn{1}{c|}{\cellcolor[HTML]{C0C0C0}Valor total} \\ \hline
\multicolumn{1}{|l|}{} &
   &
   &
   \\ \hline
\multicolumn{1}{|l|}{} &
   &
   &
   \\ \hline
\multicolumn{1}{|l|}{} &
   &
   &
   \\ \hline
\multicolumn{3}{|c|}{SUBTOTAL} &
  \multicolumn{1}{c|}{} \\ \hline
\rowcolor[HTML]{C0C0C0}
\multicolumn{3}{|c|}{TOTAL} &
   \\ \hline
\end{tabularx}%
\end{table}


\section{11. Matriz de asignación de responsabilidades}
\label{sec:responsabilidades}
\begin{consigna}{red}
Establecer la matriz de asignación de responsabilidades y el manejo de la autoridad completando la siguiente tabla:

\begin{table}[htpb]
\centering
\resizebox{\textwidth}{!}{%
\begin{tabular}{|c|c|c|c|c|c|}
\hline
\rowcolor[HTML]{C0C0C0} 
\cellcolor[HTML]{C0C0C0} &
  \cellcolor[HTML]{C0C0C0} &
  \multicolumn{4}{c|}{\cellcolor[HTML]{C0C0C0}Listar todos los nombres y roles del proyecto} \\ \cline{3-6} 
\rowcolor[HTML]{C0C0C0} 
\cellcolor[HTML]{C0C0C0} &
  \cellcolor[HTML]{C0C0C0} &
  Responsable &
  Orientador &
  Equipo &
  Cliente \\ \cline{3-6} 
\rowcolor[HTML]{C0C0C0} 
\multirow{-3}{*}{\cellcolor[HTML]{C0C0C0}\begin{tabular}[c]{@{}c@{}}Código\\ WBS\end{tabular}} &
  \multirow{-3}{*}{\cellcolor[HTML]{C0C0C0}Nombre de la tarea} &
  \authorname &
  \supname &
  Nombre de alguien &
  \clientename \\ \hline
 &  &  &  &  &  \\ \hline
 &  &  &  &  &  \\ \hline
 &  &  &  &  &  \\ \hline
\end{tabular}%
}
\end{table}

{\footnotesize
Referencias:
\begin{itemize}
	\item P = Responsabilidad Primaria
	\item S = Responsabilidad Secundaria
	\item A = Aprobación
	\item I = Informado
	\item C = Consultado
\end{itemize}
} %footnotesize

Una de las columnas debe ser para el Director, ya que se supone que participará en el proyecto.
A su vez se debe cuidar que no queden muchas tareas seguidas sin ``A'' o ``I''.

Importante: es redundante poner ``I/A'' o ``I/C'', porque para aprobarlo o responder consultas primero la persona debe ser informada.

\end{consigna}

\section{12. Gestión de riesgos}
\label{sec:riesgos}

\begin{consigna}{red}
a) Identificación de los riesgos (al menos cinco) y estimación de sus consecuencias:
 
Riesgo 1: detallar el riesgo (riesgo es algo que si ocurre altera los planes previstos)
\begin{itemize}
\item Severidad (S): mientras más severo, más alto es el número (usar números del 1 al 10).\\
Justificar el motivo por el cual se asigna determinado número de severidad (S).
\item Probabilidad de ocurrencia (O): mientras más probable, más alto es el número (usar del 1 al 10).\\
Justificar el motivo por el cual se asigna determinado número de (O). 
\end{itemize}   

Riesgo 2:
\begin{itemize}
\item Severidad (S): 
\item Ocurrencia (O):
\end{itemize}

Riesgo 3:
\begin{itemize}
\item Severidad (S): 
\item Ocurrencia (O):
\end{itemize}


b) Tabla de gestión de riesgos:      (El RPN se calcula como RPN=SxO)

\begin{table}[htpb]
\centering
\begin{tabularx}{\linewidth}{@{}|X|c|c|c|c|c|c|@{}}
\hline
\rowcolor[HTML]{C0C0C0} 
Riesgo & S & O & RPN & S* & O* & RPN* \\ \hline
       &   &   &     &    &    &      \\ \hline
       &   &   &     &    &    &      \\ \hline
       &   &   &     &    &    &      \\ \hline
       &   &   &     &    &    &      \\ \hline
       &   &   &     &    &    &      \\ \hline
\end{tabularx}%
\end{table}

Criterio adoptado: 
Se tomarán medidas de mitigación en los riesgos cuyos números de RPN sean mayores a ....

Nota: los valores marcados con (*) en la tabla corresponden luego de haber aplicado la mitigación.

c) Plan de mitigación de los riesgos que originalmente excedían el RPN máximo establecido:
 
Riesgo 1: Plan de mitigación (si por el RPN fuera necesario elaborar un plan de mitigación).
  Nueva asignación de S y O, con su respectiva justificación:
  - Severidad (S): mientras más severo, más alto es el número (usar números del 1 al 10).
          Justificar el motivo por el cual se asigna determinado número de severidad (S).
  - Probabilidad de ocurrencia (O): mientras más probable, más alto es el número (usar del 1 al 10).
          Justificar el motivo por el cual se asigna determinado número de (O).

Riesgo 2: Plan de mitigación (si por el RPN fuera necesario elaborar un plan de mitigación).
 
Riesgo 3: Plan de mitigación (si por el RPN fuera necesario elaborar un plan de mitigación)

\end{consigna}


\section{13. Gestión de la calidad}
\label{sec:calidad}

\begin{consigna}{red}
Para cada uno de los requerimientos del proyecto indique:
\begin{itemize} 
\item Req \#1: Copiar acá el requerimiento.

Verificación y validación:

\begin{itemize}
\item Verificación para confirmar si se cumplió con lo requerido antes de mostrar el sistema al cliente:\\
Detallar 
\item Validación con el cliente para confirmar que está de acuerdo en que se cumplió con lo requerido:\\
Detallar  
\end{itemize}

\end{itemize}

Tener en cuenta que en este contexto se pueden mencionar simulaciones, cálculos, revisión de hojas de datos, consulta con expertos, etc.

\end{consigna}

\section{14. Comunicación del proyecto}
\label{sec:comunicaciones}

\begin{consigna}{red}
El plan de comunicación del proyecto es el siguiente:
\end{consigna}

% Please add the following required packages to your document preamble:
% \usepackage{graphicx}
% \usepackage[table,xcdraw]{xcolor}
% If you use beamer only pass "xcolor=table" option, i.e. \documentclass[xcolor=table]{beamer}
\begin{table}[htpb]
\centering
\resizebox{\textwidth}{!}{%
\begin{tabular}{|c|c|c|c|c|c|}
\hline
\rowcolor[HTML]{C0C0C0} 
\multicolumn{6}{|c|}{\cellcolor[HTML]{C0C0C0}PLAN DE COMUNICACIÓN DEL PROYECTO}           \\ \hline
\rowcolor[HTML]{C0C0C0} 
¿Qué comunicar? & Audiencia & Propósito & Frecuencia & Método de comunicac. & Responsable \\ \hline
                &           &           &            &                      &             \\ \hline
                &           &           &            &                      &             \\ \hline
                &           &           &            &                      &             \\ \hline
                &           &           &            &                      &             \\ \hline
                &           &           &            &                      &             \\ \hline
\end{tabular}%
}
\end{table}

\section{15. Gestión de Compras}
\label{sec:compras}

\begin{consigna}{red}
En caso de tener que comprar elementos o contratar servicios:
a) Explique con qué criterios elegiría a un proveedor.
b) Redacte el Statement of Work correspondiente.
\end{consigna}

\section{16. Seguimiento y control}
\label{sec:seguimiento}

\begin{consigna}{red}
Para cada tarea del proyecto establecer la frecuencia y los indicadores con los se seguirá su avance y quién será el responsable de hacer dicho seguimiento y a quién debe comunicarse la situación (en concordancia con el Plan de Comunicación del proyecto).

El indicador de avance tiene que ser algo medible, mejor incluso si se puede medir en \% de avance. Por ejemplo,se pueden indicar en esta columna cosas como ``cantidad de conexiones ruteadeas'' o ``cantidad de funciones implementadas'', pero no algo genérico y ambiguo como ``\%'', porque el lector no sabe porcentaje de qué cosa.

\end{consigna}

\begin{table}[!htpb]
\centering
\begin{tabularx}{\linewidth}{@{}|X|X|X|X|X|X|@{}}
\hline
\rowcolor[HTML]{C0C0C0} 
\multicolumn{6}{|c|}{\cellcolor[HTML]{C0C0C0}SEGUIMIENTO DE AVANCE}                                                                       \\ \hline
\rowcolor[HTML]{C0C0C0} 
Tarea del WBS & Indicador de avance & Frecuencia de reporte & Resp. de seguimiento & Persona a ser informada & Método de comunic. \\ \hline
 &  &  &  &  &  \\ \hline
 &  &  &  &  &  \\ \hline
 &  &  &  &  &  \\ \hline
 &  &  &  &  &  \\ \hline
 &  &  &  &  &  \\ \hline
\end{tabularx}%
%}
\end{table}

\section{17. Procesos de cierre}    
\label{sec:cierre}

\begin{consigna}{red}
Establecer las pautas de trabajo para realizar una reunión final de evaluación del proyecto, tal que contemple las siguientes actividades:

\begin{itemize}
\item Pautas de trabajo que se seguirán para analizar si se respetó el Plan de Proyecto original:
 - Indicar quién se ocupará de hacer esto y cuál será el procedimiento a aplicar. 
\item Identificación de las técnicas y procedimientos útiles e inútiles que se utilizaron, y los problemas que surgieron y cómo se solucionaron:
 - Indicar quién se ocupará de hacer esto y cuál será el procedimiento para dejar registro.
\item Indicar quién organizará el acto de agradecimiento a todos los interesados, y en especial al equipo de trabajo y colaboradores:
  - Indicar esto y quién financiará los gastos correspondientes.
\end{itemize}

\end{consigna}


\end{document}
